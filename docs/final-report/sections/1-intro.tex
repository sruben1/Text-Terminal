\section{Introduction}\label{sec:intro}

% Introduction: Introduce the problem you are trying to solve and the necessary context, motivate why solving this
% problem is important, highlight the significance of solving this problem. (What are you trying to do and why?)

Text editors are an essential tool that is quite often taken for granted since pretty much every single operating system has one included by default. Despite this, quite a lot of software engineering is required to develop a software product that is able to fulfill the requirements posed by the tasks we need and wish to perform in our text editors. This is especially the case when features like word and line count should work well with very large files. \\
Since these aspects intrigued us and we wanted to try and develop our own solution from the ground up, we choose to develop a full text editor as our OS course project. More specifically, we have opted to try and develop a plain text (file) editor with good handling of large (even multiple gigabytes large) files, since this is one of the major weak points we identified with other text editors like Visual Studio Code.
Additionally: 
\begin{itemize}
\item  It should support UTF-8 encoding since it is essential when writing a text in German and French.
\item It should be compatible with the three most common line break standards: \verb|\n| (Linux), \verb|\r\n| (Windows), \verb|\r| (Classic Mac OS). 
\item The user interface should be a bit easier to use than pure keyboard text editors (e.g. vi/vim). It should allow to use the mouse for the most important operations.
\item Finally, the user interface should be responsive even when operating on large files.
\end{itemize}

