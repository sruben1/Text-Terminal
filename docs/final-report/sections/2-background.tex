\section{Background}\label{sec:bg}
% Introduces relevant background knowledge to understand your work (e.g., existing algorithms,
% protocols, software, and technologies).

After researching possible approaches for storing the text, we settled on implementing our own \textit{piece table} data structure inspired by C. Crowley's discussion of text editor data structures \cite{crowley1998data}. The general idea of \textit{piece tables} is to store a sequence of \textit{piece descriptors}, which point to contiguous text spans in a buffer. By using a separate \textit{file buffer} for the file content and an append-only \textit{add buffer} for new content, the complexity of editing text is essentially reduced to updating this sequence of piece descriptors. Thus, by using memory mapping for the \textit{file buffer}, the size of the data structure only grows with the number of edits rather than with the file size. This makes \textit{piece tables} an excellent choice for a text editor intended to handle large files.
%GUI
\par
\smallskip
For building the text-based user interface we use ncursesw version 6 \cite{ncursesw}, because it is a library that supports various terminals. We specifically chose the wide character version of ncurses to support Unicode and international character sets (standard ncurses only has ASCII support).
\par
\smallskip
We also use xclip \cite{xclip} for our copy and paste functionality, which allows accessing the clipboard of the X11 windowing system. Although X11 is increasingly being replaced by Wayland on Linux systems, we have found that xclip remains compatible with most Wayland environments thanks to XWayland compatibility.