\section{Results}\label{sec:results}
% Show the result/capabilities of your solution through plots, tables, screenshots, videos, etc.

Overall we are very happy with the performance of our text structure. We ran different performance metrics with a 15 KB, a 20 MB, and a 1.9 GB test file. As is visible below, we managed to implement all real time features so that all important metrics stay under 1 or 2 milliseconds, which we deem very good performance, especially since it stays in this range even with text files that are gigabytes large.
\\In figure \ref{fig:renderMetric} we show different insertion patterns for the most significant cases in the 20 MB file. In this context, optimized insert signifies that contiguous inserts in the text sequence get merged into the same piece descriptor, allowing to reduce the size of the linked list in this common usage pattern (most often, text is written one char after another in our experience). We only included the metric of the 20 MB file since we didn't see a different behavior when comparing them across the three file sizes.
\begin{figure}[H]
    \centering
    \includegraphics[width=0.6\textwidth]{./images/Profiler-Render-Metric-Size1.jpg}
    \caption{gui display delay in (20 MB file)}
    \label{fig:renderMetric}
\end{figure}
In figure \ref{fig:insertVsSize01} and \ref{fig:insertVsSize02} the average of these three insertion patterns (at each insert position) is used to display a slowdown analysis between the three file sizes mentioned above. What is to note here, is that the performance seems to be more limited by other system factors while our text structure maintains similar insertion performance across all file sizes (see average variations from one run to the other).
\begin{figure}[H]
\centering
\begin{subfigure}[b]{0.48\textwidth}
    \centering
    \includegraphics[width=\textwidth]{./images/Profiler-insert-slowdown-01.png}
    \caption{a first metrics run}
    \label{fig:insertVsSize01}
\end{subfigure}
\hfill
\begin{subfigure}[b]{0.48\textwidth}
    \centering
    \includegraphics[width=\textwidth]{./images/Profiler-insert-slowdown-02.png}
    \caption{a seccond metrics run}
    \label{fig:insertVsSize02}
\end{subfigure}
\caption{ }
\end{figure}

Figure 5 depicts the time taken to determine the relevant nodes for the deletion of text that spans a growing number of nodes (piece descriptors). Even for a large amount of nodes this still performs in the range of milliseconds. At the same time, we can see that undo and redo are not affected by the amount of nodes that the deletion effects, because they work locally around the first and last node of the deleted span (these nodes are directly accessible on the undo stack).

\noindent
\\In figure 6 we can see the effect of caching the last line result when iteratively searching a text for the next occurrence of a periodically appearing pattern. Without caching, in each iteration the line number has to be determined starting from the beginning, which takes longer the bigger the position of the search result is. By using the last result in the cache, this behavior is prevented.

\begin{minipage}[t]{0.48\textwidth}
\centering
\begin{figure}[H]
\centering
\includegraphics[width=\textwidth]{./images/Profiler-Undo-Redo.jpg}
\caption{ }
\label{fig:undoMetric}
\end{figure}
\end{minipage}
\hfill
\begin{minipage}[t]{0.48\textwidth}
\centering
\begin{figure}[H]
\centering
\includegraphics[width=\textwidth]{./images/Profiler-Find-Caching.jpg}
\caption{ }
\label{fig:findMetric}
\end{figure}
\end{minipage}

%GUI
\begin{figure}[h]
    \centering
    \includegraphics[width=1.0\linewidth]{figures/results_terminal.png}
    \caption{GUI of our text editor running on WSL}
    \label{fig:GUIterminal}            
\end{figure}
\noindent
In figure \ref{fig:GUIterminal} you can see our GUI. From the top left you can see 5 lines written, one of which is blank.
\\'a Test' is marked and can be used to copy/paste or delete this section.
Ln 5-5 means marked are rows from 5 to 5, same for Col but column 1 to 7.
'Line breaks: LINUX' means that the current line break style used is LINUX ($\backslash$n).
\\Next to it, you also have total line and total word count and to the right is the short-cut to exit the editor.
At the very bottom are buttons that you can press to use them. S\&R means search and replace.
\\Many aspects of our GUI will be shown in this video \cite{demo}, as some of these things are impractical to show on a picture.